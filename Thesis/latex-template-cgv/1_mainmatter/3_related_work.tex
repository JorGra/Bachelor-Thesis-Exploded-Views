\iflanguage{ngerman}
{\chapter{Verwandte Arbeiten}}
{\chapter{Related Work}}
\label{sec:related}

% Describing the research field with relevant works on the same or similar issues.

% Aufzeigen, wer sich bereits mit dem Thema oder ähnlichen verwandten
% Themen auseinandergesetzt hat, welche Lösungswege beschrieben wurden und was die Verbindung der jeweiligen zur eigenen Arbeit ist.
% Kurzen ausblick auf die eigene Arbeit und was man neu gemacht hat. Own Contributions


Occlusion avoidance is a widely researched area of computer visualization that is continuously evolving with the technology. 
New devices and technologies enable innovative ways to extend and improve known concepts and methods. 
This is especially true for visualization techniques such as exploded views and cutting planes, which benefit greatly from these novel interaction possibilities.
 
For example, Li et al. describe a method in their paper that automatically separates drawn explosion views to make them interactive. %TODO cite=li2004interactive
Their algorithm takes 2D images of explosion view diagrams and automatically cuts them apart, both reducing visual clutter and clarifying the spatial separation of the individual components. %TODO add picutres from paper
Furthermore, the separated parts can be better labeled and retracted and extended as desired.

Since two-dimensional images are limited in their interaction possibilities and the data sets and models have become increasingly complex, the question arises as to how the same principles of explosion views can be transferred to three-dimensional objects.
For this purpose, Mohammad et al developed a tool that creates exploded views for three-dimensional CAD models. It shows both precise spatial relationships and the order in which the object was assembled. % TODO cite=Mohammad_1993
This is especially useful for the visualization of machines and gives the viewer a clear idea of the arrangement. 
A disadvantage of their implementation is that the individual relations must be clearly defined by a designer beforehand to enable the generation of the explosion view and to calculate the position of the exploded parts.
So the order of composition and the blocking elements must be known and have to be defined manually. 

Li et al therefore presented a system that automatically extracts non-blocking exploded views from a 3D model, focusing on rearranging parts instead of hiding obscuring geometry. %TODO cite=Wilmot_Li_2008
They also provide a list of tools to interact with the exploded views and dynamically select and show parts of interest.
Their implementation works for both hierarchical and non-hierarchical models, which also allows it to process biological datasets where there is no fixed assembly order.  %TODO add pictures
The algorithm works by calculating an explosion graph when loading the model, which describes the blocking elements of each part from different angles. 
This allows to retrieve at runtime the sequence of elements needed to disassemble the object without parts passing through others. 
Thus a dynamic explosion graph can be generated which shows an animated composition from all viewing directions.
An important part of this is the generation of a correct explosion graph. 
To accomplish this, two problems have to be solved: first, how to move the parts to uncover the target parts without occluding them, and second, how to deal with enclosed parts. 
Li et al. solve this problem by iteratively going through all the parts and testing for two conditions: each part must be moved so that none of the target parts are obscured; if the part is a target part, it must not be obscured by any part that has already been visited. 
In order to isolate target parts from other touching parts, it is also made sure that they are completely visible and close parts are moved further away. 
If one part is completely enclosed by another, the outer one is separated in the bounding box center and pulled apart so that the inner parts are completely visible, then the algorithm continues. %TODO add pictures
The resulting application generates animated exploded views for models with up to fifty parts. 
However, a disadvantage of this implementation is that it only works for static data sets and does not provide any solution for time-varying data sets.

Tatzgern et al. improve on the work of Li et al. by finding frequently recurring subsets in a mesh and grouping them, then selecting the best representative of that group and exploding it based on a quality score.  %TODO cite=Tatzgern_2010
The frequently recurring subsets are found automatically based on a frequent sub-graph search (FSG). %TODO add pictures
The resulting explosion diagrams are especially useful for technological models where there are many identical subsets, and the explosion displays only one of them instead of doing this for each of these subsets and taking up a lot of screen space.  
For biological datasets, however, this extension is less useful, since it brings little advantage due to the distinct structure of biological objects.
More relevant, however, is its quality score which is used to select the representative. This is also applicable to general explosion views and can be used to quantitatively describe the quality of an explosion view. It is defined by the following evaluation criteria: 


\begin{itemize}
	\item \textbf{Size of the footprint of the exploded view:} Describes the entire screen space that the exploded view occupies.
	\item \textbf{Visibility of parts of the exploded group:} Describes the relative measure for the general visibility of the parts.
	\item \textbf{Part directions relative to current camera viewpoint:} Assumes that explosions similar to viewing direction are more difficult to read, they compute the average dot product between the viewing angle and the explosion direction.
	\item \textbf{Size of footprint of all other similar groups without any displacements:} Describes how well other similar groups are visible when selected representative is exploded.
\end{itemize}

These criteria are then weighted by Tatzgern et al, which influences the selection of the representative.
Even if not all of these points are suitable for use in virtual reality, some ideas can still be applied. In particular, using the dot product between the camera position and the direction of the exploding parts is a helpful approach.

While Li et al. and Tatzgern et al. relied on calculating the position of the exploded parts, Bruckner et al. used various \textbf{force-based techniques} to transform parts to generate explosion diagrams. %TODO cite=Bruckner_2006
Their presented method works with volumetric data sets and splits them into pieces before exploding. 
For this purpose, they present three tools that interactively split the dataset at runtime using split axes and cutting planes. 
The first tool splits the first object hit by a ray based on the camera's viewing direction, the second splits all objects not just the first, while the third allows the user to draw a line onto which all parts are projected, if the projection lies on the line the part is split. 
These methods can also be applied to voxel data sets and are therefore also suitable for biological objects.
Another interesting approach they use is the definition of forces acting on all parts to explode the object.
These forces are defined by Bruckner et al. as follows:

\begin{itemize}
	\item \textbf{Return force:} The part should move as little as possible from its original position. Therefore, there is a force pushing the part back to its original position. Bruckner et al use the following formula, where $r$ is the vector from the current vertex position to that of its original and $c_r$ is a constant factor. 
	\begin{equation}
		F_r = c_r * ln(\|r\|) * \frac{r}{\|r\|}
		\label{eq:returnforce}
	\end{equation}
	
	\item \textbf{Visibility of parts of the exploded group:} Describes the relative measure for the general visibility of the parts.
	\item \textbf{Part directions relative to current camera viewpoint:} Assumes that explosions similar to viewing direction are more difficult to read, they compute the average dot product between the viewing angle and the explosion direction.
	\item \textbf{Size of footprint of all other similar groups without any displacements:} Describes how well other similar groups are visible when selected representative is exploded.
\end{itemize}










